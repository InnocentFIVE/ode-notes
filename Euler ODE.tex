\documentclass[UTF8,9pt]{article}
\usepackage{ctex}
\usepackage[utf8]{inputenc}
\usepackage{amssymb}
\usepackage{amsfonts}
\usepackage{amsmath}
\usepackage{upgreek}
\usepackage{graphicx}
\usepackage[a4paper,scale=0.75]{geometry}
\usepackage{unicode-math}
\usepackage{tcolorbox}
\usepackage{fontspec}
\usepackage{tikz}
\usepackage{extarrows}
\usepackage{pxrubrica}
\usepackage{fancyhdr}
\usepackage{lipsum}
\usepackage{ulem}
\usetikzlibrary{patterns}
\tcbuselibrary{skins} 
\tcbuselibrary{breakable}
\pagestyle{fancy}
\setmainfont{EB Garamond}[Ligatures=Rare]
\setmathfont{Garamond-Math.otf}[StylisticSet={2, 7, 9}]
\setmathfont{Garamond-Math.otf}[StylisticSet={2, 7, 9, 8}, version=a]
\setmathfont[range = "0211C]{Latin Modern Math}
\renewcommand{\symcal}[1]{{\mathversion{a}\mbox{$\symscr{#1}$}}}
\xeCJKDeclareSubCJKBlock{symbolCJK}{"2018 , "201C , "300C , "300E , "3014 , "FF08 , "FF3B , "FF5B ,"3008 , "300A , "3016 , "3010 ,"2014 , "2026 , "3001 , "3002 , "FF0C , "FF0E , "FF1A , "FF1B ,"FF01 ,"FF1F , "FF05 , "3015 , "FF09 , "FF3D , "FF5D , "3009 ,"300B , "3017 , "3011 , "2019 , "201D , "300D , "300F}
\setCJKmainfont[symbolCJK=Source Han Serif SC]{GenWanMin TW TTF}
\everymath{\displaystyle}
\begin{document}
\setlength{\lineskip}{5pt}
\setlength{\lineskiplimit}{2.5pt}


\section{Euler 方程}

Euler方程是一种特殊的线性微分方程:
\[
\symscr{E}_n(x)\coloneq \sum\limits_{i=1}^{n} a_it^i\symscr{D}_t ^i(x)=0
.\]

实际上存在初等解法,设 \(y=\ln t\) ,则:
\[
\begin{aligned}
t\symscr{D}_t (x) &=t\symscr{D}_y (x) \symscr{D}_t (y)=t\symscr{D}_y(x)\frac{1}{t},\\ 
t^2 \symscr{D}^2_t (x) &= t^2 \symscr{D} _t(  \symscr{D}_t (x)) = t^2 \symscr{D} _t\left(  \frac{ \symscr{D}_y(x)}{t} \right) =-\symscr{D}_y (x)+t\symscr{D} _y^2(x)\symscr{D} _t(y) =\symscr{D} _y^2(x)-\symscr{D} _y(x).
\end{aligned}
\]
事实上有:
\[
   t^{n} \symscr{D}^n_t (x)  = \sum\limits_{i=0}^{n-1} (\symscr{D} _y-i)(x) =\symscr{D}_y (\symscr{D}_y-1)\dotsm (\symscr{D}-n+1)x
.\]
此处可以用数学归纳法证明,但是非常繁琐,此处从略。

因此我们将 \(\symscr{E}_n(x)\) 转变成了一个同阶的常系数线性微分方程,因此可以按照方法解之得到 \(y\) ,最后按照 \(y=\ln t\) 代入即可。

\end{document}