\let\symcal\undefined
\DeclareRobustCommand{\symcal}[1]{{\mathpalette\youwillneverusethisnamewhichaboutchi{#1}}}
\newcommand{\youwillneverusethisnamewhichaboutchi}[2]{\mbox{\mathversion{a}$#1\symscr{#2}$}}

\titleformat{\section}[display]
{}{\newpage\thispagestyle{forsection}
	\filright{\hrule width 1cm\vskip-1.95em}\hspace*{1cm}
	\enspace
	{\LARGE\char"E001}
	\enspace
	\raisebox{0.8ex}{$\underset{\mbox{\tiny\textit{SECTION}}}{\textbf{\textit{\Huge{\arabic{section}}}}}$}
	\enspace{\LARGE\char"E002}
	\enspace
	\titlerule}{1em}{\vspace{-3ex}\Large\bfseries\rightline}

\titleformat{\subsection}[hang]
{}{\normalsize\filright\raisebox{-1.3ex}{\huge\char"B6}\kern-2.3ex{\color{white}\normalsize\arabic{subsection}}\enspace}{0.5em}{\normalsize\bfseries}

\pagestyle{empty}
\setlength{\lineskip}{5pt}
\setlength{\lineskiplimit}{2.5pt}

\setcounter{section}{-1}

\section*{前言}

这篇文章(如果写得完的话)应该是一篇 ODE 的小短文,按照笔者的想法,笔者更倾向于将叙述形式变为两个人之间的对话,这两个人就叫做“朱鹭子”和“琪露诺”吧(笑),至于为什么……那只是因为笔者喜欢罢了(笑)。

由于笔者才疏学浅,其中必定错漏百出,请多指教。

\rightline{\ruby[g]{小飞舞}{Innocent\ FIVE}}
\newpage
\tableofcontents
\newpage
\phantom{2}
\vspace{2em}
\begin{spacing}{2}
	\rightline{\Huge\parbox{1em}{\fontsize{60pt}{80pt}{\bfseries 朱鹭子和琪露诺}}\hspace{2em}}
\end{spacing}
\clearpage
\pagestyle{fancy}

