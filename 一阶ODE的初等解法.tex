\section{一阶ODE的初等解法}
朱鹭子:一阶的ODE在某种意义上是最简单的ODE了,所以我们可以先来研究它。我们知道,ODE事关待求解函数 \(x(t)\) 的导数,这个导数可能是很高阶的,比如:

\[
	\symscr{D} ^n_t (x)=f(t,x)
	.\]

此时我们……

琪露诺:等等等等,这个 \(\symscr{D} \) 是什么……

朱鹭子:哦这个啊,我们管它叫“求导算符”,这里这样写指的是对 \( t\) 求 \(n\)次导,同时,我们也默认 \(x\) 是关于 \(t\) 的函数。我么继续,上面这个式子是一个 \(n\)阶的ODE,我们不想讨论它,我们直接来看最简单的:
\[
	\symscr{D} _t (x)=f(t)g(x)
	.\]
这种被称为“变量分离方程”。
\subsection{变量分离方程}
朱鹭子:这种我们可以这样做:
\[
	\dfrac{\symscr{D} _t(x)}{g(x)}= f(t)
	.\]
两边积分就可以得到:
\[
	\int \dfrac{1}{g(x)} \,\mathrm{d}x =\int f(t) \,\mathrm{d}t \tag{A}
	.\]

琪露诺:诶等等,这个 \(\symscr{D} _t(x)\) 到哪里去了?

朱鹭子:啊这个的话……实际上是有
\[
	\int \dfrac{1}{g(x)} \,\mathrm{d}x =\int \dfrac{\symscr{D} _t(x)}{g(x)} \,\mathrm{d}t
	.\]
这个,这个你不是有学的吗?

琪露诺:呃我忘了……

朱鹭子:啊这……好,现在我们看 A 方程的两边,左边只和 \(x\) 有关,右边只和 \(t\)有关,那么我们实际上就把 \(x(t)\)解出来了,虽然用的是隐式方程的表示法。

琪露诺:原来如此,这有什么要注意的地方吗?

朱鹭子:有,积分记得加上常数。接下来我们来看另外一种方程。
\subsection{线性微分方程}
朱鹭子:好,我们来看看这种:
\[
	\dx{}+p(t)x=f(t)\tag{B}
	.\]
这种被称为线性微分方程。

琪露诺:\emoji{🤔}诶……这个线性指的是什么意思呢……

朱鹭子:我们可以这样理解。假设 \(x_1\)和 \(x_2\)都是方程  B 的解,那么我们考虑:
\[
	x_3=cx_1+(1-c)x_2
	.\]
这个是不是方程B的解呢?

琪露诺:\emoji{😥}好像是的。

朱鹭子:对,没错,这里实际上是有:
\[
	\begin{cases}\\[-2em]
		\symscr{D} _t^n(x_1+x_2)=\symscr{D} _t^n(x_1)+\symscr{D} _t^n(x_2), \\
		\symscr{D} _t^n(cx_1)=c\cdot \symscr{D} _t^n(x_1).
	\end{cases}\tag{C}
\]
这里指的是求导算符的函数线性,另外我们考虑更普遍的形式:
\[\label{linearofl}
	\begin{aligned}
		 & \symscr{L}_1 =\symscr{D} _t^n +a_{n-1}(t)\symscr{D} _t^{n-1}+\cdots a_1(t)\symscr{D} _t +a_0(t)=\symscr{D} _t^{n-1}+\sum_{i=1}^{n-1} a_{i}(t)\symscr{D} _t^i , \\
		 & \symscr{L}_2 =\symscr{D} _t^n +b_{n-1}(t)\symscr{D} _t^{n-1}+\cdots b_1(t)\symscr{D} _t +b_0(t)=\symscr{D} _t^{n-1}+\sum_{i=1}^{n-1} b_{i}(t)\symscr{D} _t^i.
	\end{aligned}
\]
那么算符 \(\symscr{L}_1 \)和 \(\symscr{L} _2\)满足:
\[
	\left(\symscr{L} _1+\symscr{L} _2 \right) (x)=\symscr{L} _1(x)+\symscr{L} _2 (x)
	.\]
这被称为算符线性。

因此合并起来就是:

\[
	\left( \symscr{L} _1+\symscr{L} _2 \right) \left( x_1+x_2 \right) =\symscr{L} _1(x_1)+\symscr{L} _1(x_2)+\symscr{L} _2(x_1)+\symscr{L} _2(x_2)
	.\]

琪露诺:哦哦,原来线性指的是我可以随意拆掉括号啊……

朱鹭子:不可以哦,只是满足C方程罢了,你只能在规定的范围内行动。虽说如此,线性方程确实是比较简单的一类。好,我们来看B方程,这个怎么解呢?

琪露诺:我在寺子屋里见过这些题,一般来说左边都是一些函数的导数,然后两边积分就可以求出函数出来。

朱鹭子:没错,但是在这里,这个 \(p(x)\) 是如此随机,左边很可能不是个新函数的导数,我们应该怎么办呢?

琪露诺:\emoji{😭}呜哇哇……

朱鹭子:事实上,我们可以乘上一个新的函数,改变左边那一坨的结构:
\[
	B \implies h(t)\dx{}+h(t)p(t)x=f(t)h(t)\tag{D}
	.\]

琪露诺:可是这样左边还是很怪啊,看起来根本不像个导数……

朱鹭子:是的,所以我们硬来,考虑新的函数 \(h(t)x\),那么它的导数就是:
\[
	\symscr{D} _t\left( h(t) x(t)  \right)= h(t) \dx +\symscr{D} _t(h(t))x
	.\]
这样我们观察D式,我们可以强行要求:
\[
	h(t)p(t)=\symscr{D} _t\left( h(t) \right)
	.\]

琪露诺:看起来像那个巫女的作风……

朱鹭子:不是哦,\newsout{灵梦哪有那么聪明}。好了,接下来我们的任务就是解出这个奇怪的 \(h(t)\),还记得变量分离方程吗?\(h(t)\)的约束就是这个变量分离方程哦。

琪露诺:记得,所以我们只要:
\[
	h(t)p(t)=\symscr{D} _t\left( h(t) \right) \implies \int \dfrac{1}{h(t)} \,\mathrm{d}h =\int p(t) \,\mathrm{d}t
	.\]

这意味着:

\[
	\ln h(t)= \int p(t) \,\mathrm{d}t \implies h(t)=\symrm{e} ^{\int p(t) \,\mathrm{d}t }
	.\]

朱鹭子:不对哦,你积分没加常数。

琪露诺:呜哇……那加了常数之后,就是:
\[
	\ln h(t)= \int p(t) \,\mathrm{d}t +c_1\implies h(t)=\symrm{e} ^{c_1}\cdot \symrm{e} ^{\int p(t) \,\mathrm{d}t }
	.\]

是这样吗?

朱鹭子:其实你可以不加的,因为你有一个积分号没有被消掉……但先不管这个,我们可以看到 \(h\)我们已经求出来了,接下来呢?

琪露诺:那这个  \(c_1\)怎么确定啊?

朱鹭子:不用确定,都是满足条件的不是吗?为了容易算我们就让它是0好了。

琪露诺:哦哦,那这样子C方程就是:
\[
	\symscr{D} _t(h(t)x)=f(t)h(t)
	.\]

这是一个变量分离方程,我可以对它进行积分:
\[
	h(t)x= \int f(t)h(t) \,\mathrm{d}t \implies x= \dfrac{\displaystyle \int f(t)h(t) \,\mathrm{d}t}{h(t)}
	.\]

最后是代入 \(h(t)\)的方程得到:
\[
	x=\dfrac{\displaystyle \int f(t)\symrm{e} ^{\int p(t) \,\mathrm{d}t } \,\mathrm{d}t}{\symrm{e} ^{\int p(t) \,\mathrm{d}t }}
	.\]

哇,这啥啊……

朱鹭子:你做的是正确的,这就是B方程的解啦。

\subsection{常数变易法}

琪露诺:原来如此,所以这一节我完全懂了对吧。

朱鹭子:不是哦,再和你说些东西,我们回过头考虑一种更简单的情况:
\[
	\dx+p(t)x=0
	.\]

琪露诺:我知道我知道,这是分离变量方程!

朱鹭子:对,他的解是 \(x=c_1\exp \left(- \int p(t) \,\mathrm{d}t  \right) \) ,我们反倒可以这样考虑,这个 \(c_1\) 并不是一个常数,而是一个函数 \(c_1(t)\),此时 \(x=c_1(t)\exp \left(- \int p(t) \,\mathrm{d}t  \right)\)反倒是方程B的解,那么会发生什么事呢?

琪露诺:我猜猜……将这个新东西带进去之后会得到一个新的方程。

朱鹭子:没错,我们试试看:
\[
	\begin{aligned}
		 & \phantom{=}\ \,\,\dx+p(t)x=\underline{\symscr{D} _t\left( c_1(t)\exp \left(- \int p(t) \,\mathrm{d}t  \right) \right)} +p(t) c_1(t)\exp \left(- \int p(t) \,\mathrm{d}t  \right)
		\\
		 & =\underline{\symscr{D} _t(c_1(t))\exp \left(- \int p(t) \,\mathrm{d}t  \right)+c_1(t)\exp \left(- \int p(t) \,\mathrm{d}t  \right)(-p(t))}+p(t) c_1(t)\exp \left(- \int p(t) \,\mathrm{d}t  \right) \\
		 & = \symscr{D} _t(c_1(t))\exp \left(- \int p(t) \,\mathrm{d}t  \right) =f(t).
	\end{aligned}
\]

琪露诺:所以这个 \(c_1(t)\)就满足:
\[
	c_1'(t)=f(t)\exp \left(\int p(t) \,\mathrm{d}t  \right) \implies c_1(t)=\int \left[ f(t)\exp \left(\int p(t) \,\mathrm{d}t  \right)  \right]  \,\mathrm{d}t
	.\]

天……这太多积分号了叭……

朱鹭子:确实,但这玩意不是和你之前求的B方程的解差不多吗?

琪露诺:哦也是,那这个方法有什么用呢?

朱鹭子:用处多了,以后我们在解更普遍的方程会用到这个解法,这被称为\textbf{常数变易法}.

琪露诺:哦……

朱鹭子:接下来我们看些更神奇的。

\subsection{恰当微分方程}
考虑一些更通用的情况:
\[
	\dx=f(x,t)
	.\]

我们可以把它写成微分形式:

\[
	f(x,t)\,\symrm{d} t-\symrm{d} x=0
	.\]

再考虑更一般的情况:

\[
	T(x,t)\,\symrm{d} t+ X(x,t)\,\symrm{d} x=0
	.\]

琪露诺:这是啥……

朱鹭子:看到了吗?这其实是微分方程的一种特殊写法,看到这你想到了什么?

琪露诺:\emoji{🤔}……第二类线积分?

朱鹭子:没错!第二类线积分里面有些很好算的情况,那就是“路径无关”的时候,此时应该由什么判别呢?

琪露诺:\emoji{😥}由Green公式应该有:

\[
	\dfrac{\uppartial T}{\uppartial x}=\dfrac{\uppartial X}{\uppartial t}
	.\]

朱鹭子:在函数性质良好的前提下,可以这么做,那么如果是路径无关的话接下来这么求解方程呢?

琪露诺:呃……

朱鹭子:事实上,路径无关场就是梯度场,这意味着存在一个函数 \(u\) 使得:
\[
	\dfrac{\uppartial u}{\uppartial t}=T,\qquad \dfrac{\uppartial u}{\uppartial x}=X
	.\]
所以方程就变成了:
\[
	\dfrac{\uppartial u}{\uppartial t}\,\symrm{d} t+ \dfrac{\uppartial u}{\uppartial x}\,\symrm{d} x=0
	.\]

左边就是 \(u\)的全微分啦,所以我们就有:

\[
	u=c_1
	.\]

这样就解出了 \(x\)了。

琪露诺:那如果没有路径无关怎么办?

朱鹭子:你可以想一下,我们之前怎么解决不是全微分的问题的。

琪露诺:\emoji{😥}难道你说的是,乘上一个新的函数?

朱鹭子:可以这样做,看看吧:
\[
	\dfrac{\uppartial u}{\uppartial t}=hT,\qquad \dfrac{\uppartial u}{\uppartial x}=hX
	.\]

接下来呢?

琪露诺:呃……不知道了。

朱鹭子:事实上我们可以考虑Young定理:
\[
	\dfrac{\uppartial ^2u}{\uppartial x \uppartial t}=\dfrac{\uppartial ^2u}{\uppartial t \uppartial u}\implies \dfrac{\uppartial (hT)}{\uppartial x}=\dfrac{\uppartial (hX)}{\uppartial t}
	.\]

接下来我们可以用乘法法则分离它们:
\[
	X \dfrac{\uppartial h}{\uppartial t}-T \dfrac{\uppartial h}{\uppartial x}=\left( \dfrac{\uppartial T}{\uppartial x}-\dfrac{\uppartial X}{\uppartial t} \right) h
	.\]

琪露诺:好的,那这个方程怎么解?

朱鹭子:事实上这是一个偏微分方程……单纯解它比解原来的ODE更困难。

琪露诺:啊这……这不就是……寄了吗?

朱鹭子:差不多吧,不过的确有一些比较简单的情况。

\begin{itemize}
	\item 当 \( \dfrac{\dfrac{\uppartial T}{\uppartial x}-\dfrac{\uppartial X}{\uppartial t}}{X} =p(t)\) 与 \(x\) 无关的时候。我们有:
	      \[
		      h(t,x)=\exp \left( \int p(t)\,\mathrm{d}t  \right)
		      .\]
	\item 当 \( \dfrac{\dfrac{\uppartial T}{\uppartial x}-\dfrac{\uppartial X}{\uppartial t}}{T} =q(x)\) 与 \(t\) 无关的时候。我们有:
	      \[
		      h(t,x)=\exp \left(- \int q(x)\,\mathrm{d}x \right)
		      .\]
	\item 当 \( {\dfrac{\uppartial T}{\uppartial x}-\dfrac{\uppartial X}{\uppartial t}}=p(t)X-q(x)T\)的时候。我们有:
	      \[
		      h(t,x)=c_1 \exp \left( \int p(t) \,\mathrm{d}t +\int q(x) \,\mathrm{d}x  \right)
		      .\]
\end{itemize}

琪露诺:前两个还好,第三个也太那啥了,这个只有天才才能看出来吧。

朱鹭子:这个不管,反正算是比较简单的情况了(笑)。当然你也可以直接用看就把这个 \(h\)(我们管它叫积分因子)找出来。

琪露诺:这能看出来的吗?

朱鹭子:看你的直觉了……比如说这个?
\[
	\left( x^2 +y^2 +x \right) \,\symrm{d} x +y \,\symrm{d} y=0
	.\]

琪露诺:哇……看起来好难……哭……

朱鹭子:看,其实我们仔细观察的话倒是可以看出一些对偶量:
\[
	\left( x^2 +y^2  \right) \,\symrm{d} x +y \,\symrm{d} y+ x \,\symrm{d} x=0
	.\]

尝试 \(h=\dfrac{1}{x^2 +y^2 }\),我们可以得到:
\[
	\symrm{d} x+ \dfrac{x\,\symrm{d} x+y\,\symrm{d} y}{x^2 +y^2 }=0\implies \symrm{d} x+\symrm{d} \left( \dfrac{1}{2}\ln \left( x^2 +y^2  \right)  \right) =0
	.\]

所以最后的结果就是:
\[
	x+ \dfrac{1}{2}\ln \left( x^2 +y^2  \right)=c_1
	.\]

琪露诺:这根本不是人能看出来的……

朱鹭子:诶别说,这种对那只九尾策士的确是显然的题……

琪露诺:啊。所以,一阶微分方程就到这里了?

朱鹭子:不对哦,这只是一些基础的,还有一些奇怪的情况需要提及一下。那就是其解的参数表示。

琪露诺:\emoji{😥}好像确实之前没有讲关于参数方程的任何内容呢……

朱鹭子:我们来看一些隐式方程:
\[
	x=f(t,x')
	.\]

这种反倒是将 \(x\) 分离出来了,此时怎么办呢?

琪露诺:\emoji{😣}这不又是一个一般的式子吗,怎么可能会有什么通法?

朱鹭子:你想想看吧,这种和之前的有什么不同?

琪露诺:\emoji{😮}首先是 \(x'\)不能再随意地被分离出来,然后是左边是 \(x\)……

朱鹭子:差不多,想想左边有没有办法变成 \(x'\)呢?

琪露诺:啊,你是说求导?

朱鹭子:对咯:

\[
	\dx= \dfrac{\uppartial f}{\uppartial t}+ \dfrac{\uppartial f}{\uppartial \left( x' \right) }\symscr{D} _t^2(x)
	.\]

我们不妨设 \(p= \dx\),那么这个方程就可以变为:

\[
	p=\dfrac{\uppartial f}{\uppartial t}+ \dfrac{\uppartial f}{\uppartial p}\symscr{D} _t(p)
	.\]

琪露诺:可是这个方程不还是很怪吗?也不是有通解的那种……诶等等,这下好像 \(\symscr{D} _t(p)\)就分离出来了:
\[
	\symscr{D} _t(p)= \dfrac{p-{\uppartial f}/{\uppartial t}}{{\uppartial f}/{\uppartial p}}
	.\]

接下来我只要按照前面三种方法做就可以解出 \(p\),然后积分就可以得到 \(x\)了对吧?

朱鹭子:差不多是这样,但是如果你解出来 \(p\),但是却是无法分离的,导致你没法直接积分,那该怎么办呢?况且你既然能解出 \(p\),把它往 \(f\)里面一代,不就把 \(x\)解出来了吗,为什么要积分呢?

琪露诺:啊啊啊哇。

朱鹭子:最令人难受的应该是你解出来得到这个吧:

\[
	\psi (t,p,c_1)=0
	.\]

你看,你甚至连通解的参数都是嵌在函数里面的,此时你不考虑参数方程还能怎么办呢?

琪露诺:通解参数是什么哇?

朱鹭子:啊,就里面这个 \(c_1\),事实上对于一个 \(n\)阶的微分方程,它的通解将会长得像:
\[
	x=\psi (t,c_1,c_2,\cdots ,c_n)
	.\]

其中 \(c_1,c_2,\cdots ,c_n\) 相互独立,这到时候说一下唯一性定理你可能会更明白一些。

琪露诺:好吧,所以最后的参数方程就是:
\[
	\begin{cases}\\[-2em]
		\psi (t,p,c_1)=0, \\
		x=f(t,p).
	\end{cases}
\]

这样咯?那参数就是那个 \(t\)咯。

朱鹭子:\(t\)是自变量,\(p\)才是参数。你要知道 \(p\)才是我们想要消去的那个,所以 \(p\)是参数。

琪露诺:哦。所以这其实是一种“差不多解出来了但还有一步要走”的情况吧?

朱鹭子:的确,看看这种?
\[
	t=f(x,x')
	.\]

这下反倒是自变量被分离出来了。

琪露诺:麻……这是强行走成反函数的路线吗……

朱鹭子:要清楚你现在考虑的是微分结构,不要乱反函数。

琪露诺:呜……首先我左右两边对 \(t\)求导:
\[
	1=\dfrac{\uppartial f}{\uppartial x}\symscr{D} _t(x)+ \dfrac{\uppartial f}{\uppartial (x')}\symscr{D} _t^2(x)
	.\]

然后再设 \(p=\symscr{D} _t(x)\) ,这样子就行啦:
\[
	1=\dfrac{\uppartial f}{\uppartial x}p+ \dfrac{\uppartial f}{\uppartial p}\symscr{D} _t(p)
	.\]

朱鹭子:不错,这也有另外一种方法,那就是上来直接对 \(x\)求导:
\[
	\symscr{D} _x(t)=\dfrac{\uppartial f}{\uppartial x}+\dfrac{\uppartial f}{\uppartial p}\symscr{D} _x(p)
	.\]

然后由于一阶微分形式的不变性,我们可以把它改写为:
\[
	\dfrac{1}{p}=\dfrac{\uppartial f}{\uppartial x}+\dfrac{1}{p}\cdot \dfrac{\uppartial f}{\uppartial p}\symscr{D} _t(p)
	.\]

也和你写的一样哦。

琪露诺:哇塞。

朱鹭子:嗯,来点更怪的例子,如果你最后解出来是这样子的:

\[
	\begin{cases}\\[-2em]

		x'=X(c) , \\
		\ t=T(c).
	\end{cases}
\]

这种情况呢?你该怎么得到——没有其他条件了—— 那个\(x\)呢?

琪露诺:寄。我试试看,首先是:
\[
	\symscr{D} _c(x)=\symscr{D} _t(x)\symscr{D} _c(t)= X(c)T'(c)\implies \begin{cases}\\[-2em]
		x=\int   X(c)T'(c)\,\mathrm{d}c, \\
		\ t=T(c).
	\end{cases}
\]

依据微分形式不变性好像是这样的?

朱鹭子:的确。总而言之,你就得构造这个 \(\symscr{D} _c(x)\)然后对其积分,虽然得到的结果是参数方程,不过也值了不是吗?

琪露诺:听起来是这样没错啦……