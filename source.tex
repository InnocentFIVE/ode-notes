\documentclass{ctexart}
\usepackage{geometry}[b5paper,scale=0.85]
\usepackage{fontenc}
\usepackage{fontspec}
\usepackage{tcolorbox}
\usepackage{textcomp}
\usepackage[dutch]{babel}
\usepackage{amsmath,amsthm,amssymb,amstext,amsfonts,mathrsfs}
\usepackage{tabu,multirow}
\usepackage{caption}
\usepackage{tikz,pgfplots}
\usepackage{extarrows}
\usepackage{footnote}
\usepackage{transparent}
\usepackage{unicode-math}
\usepackage{fancyhdr}
\pagestyle{fancy}
\setCJKmainfont{C300-老报宋.ttf}
\setmathfont{Garamond-Math.otf}[StylisticSet={7,9,10,6}]
\setmainfont{EBGaramond-Regular.otf}
\everymath{\displaystyle}

\begin{document}
\setcounter{section}{-1}
\section{符号约定}
\section{一些一阶ODE的初等解法}
\section{存在性定理和唯一性定理}
首先考虑比较普遍的情况:
\[
	y'=f(x,y)
	.\]
此时若 \(f\) 在矩形区域 \(\widetilde{R}:\left\{ (x,y):\left| x-x_0 \right| \leqslant a,\, \left\vert y-y_0 \right\vert \leqslant b \right\} \)上连续且满足 Lipschitz 条件:
\[
	\exists L>0,\, \mathrm{s.t.}\,\forall (x,y_1),(x,y_2)\in \widetilde{R},\,  \left| f(x,y_1)-f(x,y_2) \right|\leqslant L \left\vert y_1-y_2 \right\vert
	.\]
对 I.V.P : \(y(x_0)=y_0\),\textbf{存在且仅存在}一个解在矩形区域: \(\hat{R}:\left\{ (x,y):\left| x-x_0 \right| \leqslant h,\, \left\vert y-y_0 \right\vert \leqslant b \right\}\)上成立,其中 \(h\) 满足:
\[
	h=\min\left\{ a,\frac{b}{M} \right\} ,\,M=\max\left| f(x,y) \right| \left( (x,y)\in \widetilde{R} \right)
	.\]
对方程:
\[
	y'=f(x,y),\qquad y(x_0)=y_0 \tag{\(\Omega \)}
	.\]
\begin{enumerate}
	\item 首先证明求解该I.V.P等价求解:
	      \[
		      y=y_0+\int_0^x f(t,y(t)) \,\mathrm{d}t \tag{\(\Gamma \)}
		      .\]
	      充分性:不妨设 \(\psi (x)\) 是(\(\Omega \))的解,则有:
	      \[
		      \frac{\mathrm{d}\psi }{\mathrm{d}x}=f(x,\psi )
		      .\]
	      由于连续直接积分得到:
	      \[
		      \psi (x)-\psi (0)=\int_{x_0}^x f(t,\psi (t)) \,\mathrm{d}t
		      .\]
	      必要性:\(\psi (x)\) 是 (\(\Gamma  \))的解,对 \(x\) 作微分就有:
	      \[
		      \frac{\mathrm{d}\psi }{\mathrm{d}x}=f(x,\psi )
		      .\]
	      \rightline{\(\blacksquare\)}
	\item 设立 Picard 序列:
	      \[
		      \begin{cases}
			      \\[-2.2em]\
			      \psi_0(x)=y_0 ,                                             \\
			      \psi _1(x)=y_0+\int_{x_0}^x \psi (t,\psi _0) \,\mathrm{d}t, \\
			      \vdots                                                      \\
			      \psi _1(x)=y_0+\int_{x_0}^x \psi (t,\psi _0) \,\mathrm{d}t. \\
		      \end{cases}
	      \]
	      然后是证明在 \(\hat{R}\) 上面,以下方程成立:
	      \[
		      \left\vert  \psi_n(x)-y_0 \right\vert \leqslant b,\,\forall \left\vert x-x_0 \right\vert \leqslant h
		      .\]
	\item 然后是证明 Picard 序列一致收敛“
	\item 接下来是证明存在性
	\item 最后是唯一性
\end{enumerate}
另外地,Lipschitz条件可以变成更强的对 \(y\) 偏导连续条件。
\subsection{解的延拓}
由唯一性定理,在 \(\left\vert x-x_0 \right\vert \leqslant h\)内解存在且唯一,如果 \(f(x,y)\)在 \(\hat{G}\)上面内闭满足Lipschitz条件,那这个解可以延拓到任意接近 \(\hat{G}\)的边界。
\subsection{解对初值的连续性}如果 \(f(x,y)\)满足上面那些条件,则它的唯一解 \(\psi (x,x_0,y_0)\) 对 \(x,x_0,y_0\)都是连续的,若 \(\frac{\partial f}{\partial y} \)连续,则是可微的。此处有(不加证明):
\[
	\begin{cases}
		\\[-2.2em]\
		\frac{\partial \psi}{\partial x_0}=-f(x_0,y_0)\exp \left( \int_{x_0}^x \frac{\partial f}{\partial y}\Bigg|_{y=\psi } \,\mathrm{d}t  \right) \\
		\frac{\partial \psi}{\partial y_0}=\exp \left( \int_{x_0}^x \frac{\partial f}{\partial y}\Bigg|_{y=\psi } \,\mathrm{d}t  \right)
	\end{cases}
	.\]
\section{高阶ODE}
\subsection{齐次线性微分方程的解的结构}
考虑一种普遍的情况:
\[
	\mathcal{L} [x]=0\tag{HOM}
	.\]
其中 \(\mathcal{L} \)代表线性算符:
\[
	\mathcal{L} = \sum\limits_{i=0}^{n} a_i\mathcal{D} ^i
	.\]
为了考虑函数是否有线性关系,定义 Wronsky 矩阵如下:
\[
	\mathcal{W} =\begin{bmatrix}
		x_1         & x_2         & \cdots & x_n         \\
		x_1'        & x_2'        & \cdots & x_n'        \\
		\vdots      & \vdots      &        & \vdots      \\
		x_1^{(n-1)} & x_2^{(n-1)} & \cdots & x_n^{(n-1)} \\
	\end{bmatrix}
	.\]
上面的 \(x_j\) 都是函数。

同时用线性代数的方法考虑线性相关/无关,比如多项式的基 \(1,t,t^2 ,t^3 ,\cdots ,t^{n}\)是线性无关的(废话)。
另外有关 Wronsky 行列式的一些定理:
\begin{enumerate}
    \item 如果函数 \(x_1,x_2,\cdots x_n\)在区间 \(I\)上是线性相关的,则在该区间上 Wronsky行列式恒为0:
    \begin{itemize}
        \item 证明如下:

        按照线性代数的知识,如果要线性相关则需要存在一组不全为零的常数 \(c_1,c_2,\cdots c_n\)使得:
        \[
        \sum\limits_{j=1}^{n} c_j x_j\equiv 0
        .\]
        在区间上面\textbf{恒成立},将该式连续求导 \(n-1\) 次:
        \[
            \sum\limits_{j=1}^{n} c_j x_j^{(m)}=0,\qquad m=0,1,\cdots n-1
        .\]
        这其实就是:
        \[
       \symbfcal{W}\cdot \symbfit{c}  = \symbf 0 \tag{\(\Theta \)}
        .\]
        其中 
        \[
        \symbfit c=\begin{bmatrix}
            c_1 \\    c_2 \\    \vdots \\    c_n \\\end{bmatrix}
        .\]
        由于 \(x_j\) 线性无关,因此不全为零的常数 \(\left\{ c_1,c_2,\cdots c_n \right\} \)存在,这意味着 \((\Theta )\) 这个方程存在至少两个解(还有零解),因此 \( \left\vert\mathcal{W}  \right\vert=0  \)。
        
        另外,逆定理一般不成立。
    \end{itemize}
    \item  如果函数 \(x_1,x_2,\cdots x_n\)\textbf{是(HOM) 的解},且在区间 \(I\)上是线性无关的,则在该区间上 Wronsky行列式恒不为0:
    \begin{itemize}
        \item 这个是 (HOM) 的解的条件非常重要,没有它不成立的(因为不能用唯一性定理)。证明如下:
        
        用反证法,如果 \(\mathcal{W} (t_0)=0\),那么
        \[
        \symbfcal W(t_0)\cdot \symbfit c=\symbf 0
        .\]
        有非零解,那么考虑此时的
        \[
        x=\sum\limits_{j=1}^{n} c_j x_j
        .\]
        当然是 (HOM) 的解,同时注意到:
        \[
        \begin{bmatrix}
            x(t_0) \\    x'(t_0) \\    \vdots \\    x^{(n-1)}(t_0) \\\end{bmatrix}= \symbfcal W(t_0)\cdot \symbfit c=\symbf 0
        .\]
        这意味着初始条件也确定了,同时留意到 \(x\equiv 0\) 也是满足I.V.P的,因此由唯一性得到 
\[
    \sum\limits_{j=1}^{n} c_j x_j=0
.\]
        还记得此的 \(c_j\) 不全为0,这意味着 \(x_1,x_2,\cdots x_n\)线性相关,矛盾!因此证毕。
    \end{itemize}
    \item 如果对应 \(n\) 阶的齐次线性微分方程 \(\mathcal{L} [x]=0\) ,我们找到了其 \(n\) 个线性无关解,那么这 \(n\) 个解的张成构成了方程的解空间。
    \begin{itemize}
        \item 易知这 \(n\) 个解的张成被包含于方程的解空间,同时对于张成元素 \(\symbfit x\cdot \symbfit c\)中的 \(\symbfit c\)的每一个元素函数无关,即:
        \[
        \begin{vmatrix}
            \frac{\partial x}{\partial c_1} & \frac{\partial x}{\partial c_2}& \cdots    & \frac{\partial x}{\partial c_n} \\
            \frac{\partial x'}{\partial c_1} &\frac{\partial x'}{\partial c_2} & \cdots    & \frac{\partial x'}{\partial c_n} \\
            \vdots & \vdots &  & \vdots \\
            \frac{\partial x^{(n-1)}}{\partial c_1} &\frac{\partial x^{(n-1)}}{\partial c_2} & \cdots   & \frac{\partial x^{(n-1)}}{\partial c_n} \\
        \end{vmatrix}\neq 0
        .\]
    \end{itemize}
\end{enumerate}
\section{线性微分方程组}


\end{document}