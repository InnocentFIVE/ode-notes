\documentclass[UTF8,9pt]{article}
\usepackage{ctex}
\usepackage[utf8]{inputenc}
\usepackage{amssymb}
\usepackage{amsfonts}
\usepackage{amsmath}
\usepackage{upgreek}
\usepackage{graphicx}
\usepackage[a4paper,scale=0.75]{geometry}
\usepackage{unicode-math}
\usepackage{tcolorbox}
\usepackage{fontspec}
\usepackage{tikz}
\usepackage{extarrows}
\usepackage{pxrubrica}
\usepackage{fancyhdr}
\usepackage{lipsum}
\usepackage{ulem}
\usepackage{nicematrix}
\usetikzlibrary{patterns}
\tcbuselibrary{skins} 
\tcbuselibrary{breakable}
\pagestyle{fancy}
\setmainfont{EB Garamond}[Ligatures=Rare]
\setmathfont{Garamond-Math.otf}[StylisticSet={2, 7, 9}]
\setmathfont{Garamond-Math.otf}[StylisticSet={2, 7, 9, 8}, version=a]
\setmathfont[range = "0211C]{Latin Modern Math}
\renewcommand{\symcal}[1]{{\mathversion{a}\mbox{$\symscr{#1}$}}}

\everymath{\displaystyle}
\begin{document}
\setlength{\lineskip}{5pt}
\setlength{\lineskiplimit}{2.5pt}
\ExplSyntaxOn
\makeatletter

\newcommand\phantomHeight[1]{\rule[#1]{0pt}{0pt}}
\newcommand\phantomDepth[1]{\rule[-#1]{0pt}{0pt}}

\newcommand\drawAboveDelimiter{\drawDelimiter{north}}
\newcommand\drawBelowDelimiter{\drawDelimiter{south}}

% Draw extented delimiter #2 which spreads from node (#3.west) to (#4.east).
%   This macro imitates the definition of \tikz@delimiter, the macro behind 
%   options of matrix library like "above delimiter" and "below delimiter".
% #1 = name of node anchor
% #2 = delimiter symbol, e.g., \rbrace
% #3 = name of beginning node
% #4 = name of ending node
% #5 = extra tikz options act on the delimiter
\def\drawDelimiter#1#2#3#4#5{
  \begin{tikzpicture}
    \node[outer~ sep=0pt, inner~ sep=0pt, 
      draw=none, fill=none, #5, rotate=-90] 
      at ($ (#3.#1)!.5!(#4.#1) $)
    {
      {\nullfont\pgf@process{
        \pgfpointdiff
          {\pgfpointanchor{nm - \int_use:N\g__nm_env_int - #3 -large}{west}}
          {\pgfpointanchor{nm - \int_use:N\g__nm_env_int - #4 -large}{east}}
      }}
      $\left.\vcenter{\nullfont\hrule height .5\pgf@x depth .5\pgf@x width0pt}\right#2$
    };
  \end{tikzpicture}
}

\makeatother
\ExplSyntaxOff

\[
    x_{\rm All}= \sum\limits_{i=1}^{n} x_i(t) \, \cdot \int 	\dfrac{{\scriptstyle	
    \begin{vmatrix}
\scriptstyle \scriptstyle x_1(t) & \scriptstyle x_2(t)         & \scriptstyle \cdots & \scriptstyle 0                                & \scriptstyle \cdots & \scriptstyle x_n(t)         \\[-3pt]
\scriptstyle x_1'(t)             & \scriptstyle x_2'(t)        & \scriptstyle \cdots & \scriptstyle 0                                & \scriptstyle \cdots & \scriptstyle x_n'(t)        \\[-3pt]
\scriptstyle \vdots              & \scriptstyle \vdots         & \scriptstyle \cdots & \scriptstyle \vdots                           & \scriptstyle \cdots & \scriptstyle \vdots         \\[-3pt]
\scriptstyle x_1^{(n-1)}(t)     &\!\! \!\scriptstyle x_n^{(n-1)}(t) \!\!\!& \scriptstyle \cdots & \scriptstyle 1                                & \scriptstyle \cdots &\!\! \!\scriptstyle x_n^{(n-1)}(t) \!\!\\[3pt]
\end{vmatrix}\kern-4em\raisebox{1.8em}{$\kern-8em\overset{k}{\overbrace{\phantom{zzxvxvxvxx3wixcb}}}$\kern 4.6em}}}{\symscr{W} (t)}f(t) \,\mathrm{d}t
.\]
\end{document}