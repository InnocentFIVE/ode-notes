\section{存在唯一性定理的推导思路}

蓝:存在唯一性定理是常微分方程论中非常重要的一个定理,这里旨在给出一些推导思路的提及。首先我们先从一个方程开始引入:
\[
	x=\cos x
	.\]

这个的解是什么?

琪露诺:这个不明显是超越方程嘛……怎么可能说得上来?

蓝:那的确也是,事实上它的解是一个暂无初等表达的无理数。

琪露诺:但这不是一个普通的方程么,和ODE有什么联系?

蓝:(笑)先别急嘛,我在给你看点东西,我构造一个序列 \(\left\{ x_n \right\}^\infty_0 \),满足:\(x_0=0,\,x_n=\cos  x_{n-1}\),然后我们来看看这个序列会怎么变化:

\begin{figure}[htp]
	\centering
	\begin{tikzpicture}
		\tikzset{>=latex}
		\pgfplotsset{width=12cm,height=7cm}
		\begin{axis}[
				axis x line=middle,
				axis y line=middle,
				every inner x axis line/.append style={->},
				every inner y axis line/.append style={->},
				legend pos = north west,
				ylabel=$x_n$,
				xlabel=$n$,
				grid style=dashed,
				samples=1000,
				xmin=0,xmax=31,ymin=0.5,ymax=1.1,
			]
			\addplot[
				domain=min: max,
				color=black,
				smooth
			]
			coordinates{
					(0,1)
					(1,0.540302)
					(2,0.857553)
					(3,0.65429)
					(4,0.79348)
					(5,0.701369)
					(6,0.76396)
					(7,0.722102)
					(8,0.750418)
					(9,0.731404)
					(10,0.744237)
					(11,0.735605)
					(12,0.741425)
					(13,0.737507)
					(14,0.740147)
					(15,0.738369)
					(16,0.739567)
					(17,0.73876)
					(18,0.739304)
					(19,0.738938)
					(20,0.739184)
					(21,0.739018)
					(22,0.73913)
					(23,0.739055)
					(24,0.739106)
					(25,0.739071)
					(26,0.739094)
					(27,0.739079)
					(28,0.739089)
					(29,0.739082)
					(30,0.739087)
				}
			;
		\end{axis}
	\end{tikzpicture}
\end{figure}


可以看到这样一个序列存在一个极限 \(X\),这个极限必然满足:

\[
	X=\cos X
	.\]

莉格露:只要对方程 \(x_n=\cos x_{n-1}\)左右两边求极限就可以了吧?

蓝:没错,这其实引申出一个不动点的问题,如果说 \(f(x)=\cos(x)\),那么 \(X\)就是 \(f\)的不动点。

橙:那这个实际上和ODE有什么关系呢?

蓝:事实上也并不是有直接关系,但是倒是可以引申出一些问题,比如说如果是这样子的方程:
\[
	x= \cos x,\,x\in\symbb{Q}
	.\]

那么方程还会有解吗?

橙、琪露诺、露娜:没有。

蓝:没错,因为它\textbf{唯一}的解并不是有理数,然而我们换一种视角能够发现,如果设:
\[
	\delta =x-\cos x,\,x\in\symbb{Q}
	.\]

那 \(\delta \) 可以随着 \(x\) 的巧妙选取变得任意小,但是不能为0,那么实际上我们就知道了,是取极限这个操作引起了质变。那为什么我们在有理数域中不能随意取极限,在实数中反而可以呢?

橙:我大概知道是什么感觉,但就是说不上来,也许是因为有理数并没有实数那样稠密吧?

蓝:嗯……实际上也并不是这样,这里有另外一个与上面那个方程没有关系的例子,一个序列 \({s_n}^\infty_0\)满足:
\[
	s_n= 10^n \left\lfloor \frac{\sqrt{2}}{10^n} \right\rfloor
	.\]

实际上就是 \(1,1.4,1.41,1.414\cdots \),即 \(\sqrt{2}\)的 \(n\)位精度下取整。那这个序列的极限显然是 \(\sqrt{2}\),并不在有理数的范围内。

莉格露:所以你的意思是说,这种现象并不是特例……而是有理数本身的问题。

蓝:的确,事实上按照外界的说法的话,就是有理数域并不是完备的赋范空间。

橙:啥?

蓝:这样吧,我们知道,我们要求得有理数之间的距离用的是让两个有理数相减的方法,但事实上,这个距离可以更为抽象,只要满足一下三点就可以:


\begin{enumerate}\kaiti
	\item 两点之间的距离大于等于零: \(\symcal d(A,B)\geqslant 0\),\(\symcal d(A,B)=0  \iff  A=B\)
	\item \(A\)到 \(B\) 的距离等于 \(B\) 到 \(A\)的距离:\(\symcal d(A,b)=\symcal d(b,A)\)
	\item 满足三角不等式:\(\symcal d(A,B)\leqslant \symcal d(A,C)+\symcal d(C,B)\)
\end{enumerate}

琪露诺:就这吗?这不就是一般的距离?

蓝:不是这样子的,只要满足上面三条公理就可以哦,比如说我们在更奇怪的东西上面定义距离:定义两个函数的距离:

\[
	\symcal d(f(x),g(x))= \max \left\vert f(x)-g(x) \right\vert
	.\]

当然这个函数是 \(\symbb{R} \mapsto\symbb{R} \)的,这样也算一种距离嘛……虽然看起来很怪。

橙:原来如此……所以人们这样子做是为了研究更广泛的东西。

蓝:没错!但是还有一些问题,即使定义了距离还是不能解决所有问题嘛,我们还有一种比较特殊的‘距离’,也就是模。这种东西可不是和0之间的距离。

莉格露:诶不是吗?

蓝:事实上,模,或者说范数,也是依靠公理来定义的,它针对的是向量,因为只有向量我们才会叫模(笑)。而函数也是向量嘛。范数的那些公理是;
\begin{enumerate}
	\item \(\left\| \symbfit x \right\|\geqslant 0\),
	\item \(\left\| \symbfit x \right\| = 0  \iff x=\symbf 0\),
	\item \(\left\| \lambda \symbfit x\right\| = \left| \lambda  \right| \cdot  \left\| \symbfit x \right\|\),
	\item \(\left\|\symbfit x_1+\symbfit x_2 \right\| \leqslant  \left\|\symbfit x_1 \right\|+ \left\|\symbfit x_2 \right\|\).
\end{enumerate}

露娜:所以一个比较好的东西是要有距离和模的,那这个完备是什么情况?

蓝:那个东西就叫做\textbf{赋范空间}了,我们知道,如果这东西都叫做空间了,那肯定对加, 数乘封闭,而完备指的是对极限运算的封闭,正如我刚才举的有理数例子一样,我对它取极限反而是不收敛的(因为根本不在 \(\symbb{Q} \)里),所以有理数域是不完备的。完备的空间有很多好的性质。比如说接下来要介绍的不动点定理。

莉格露:你指的是因为完备,所以才能取极限,所以才能通过迭代取不动点吗?

蓝:首先要有这么一个点吧。事实上,这个不动点定理涉及到压缩映射。有一个非常好的比喻就是,如果你手里拿着幻想乡的地图,那么幻想乡中必有一点与地图上对应的一点重合。

橙:好像是这样的……

露娜:但是这样必须要求地图在幻想乡内部吧。

蓝:的确,还有其他要求是地图必须要比幻想乡小——当然这是自然的。我们可以定义这个“将幻想乡变成幻想乡地图”的操作认为是一个映射,从三维空间(幻想乡)到三维空间(地图)的映射(因为地图也是三维空间中的)。实际上我们可以把这个“三维空间”缩小到幻想乡,那就是一个从幻想乡到幻想乡的映射:
\[
	M(P):\symbb{G} \mapsto\symbb{G}
	.\]

这里的 \(M\)就代表绘制地图的行为, \(P\)是空间上一点,\(\symbb{G} \)是幻想乡,实际上是三维空间的子集。这里就是声明了一个从幻想乡上的点映到幻想乡上的点的映射。

露娜:好怪……为什么是幻想乡到幻想乡而不是幻想乡到地图……

蓝:引入地图这个东西只会变得麻烦,由于地图必然在幻想乡里面,我们可以将其扩充至幻想乡(实际上也被称为“上域”),这样子就可以在更少的空间中讨论问题。接下来是压缩映射的\textbf{压缩性质}。

橙:可是地图本身就在幻想乡里就有压缩性质了哇。

蓝:没这回事,我们要的是\textbf{压缩},实际上可以与有没有在幻想乡里没有关系……如果除去这个“地图在幻想乡里面”这个性质,那压缩用什么定义?

莉格露:只要地图的直径小于幻想乡的就可以了吧?

蓝:那直径怎么定义呢?

莉格露:地图上任意两点之间距离的最大值。

蓝:看,这不是就用到距离了嘛,由于地图和幻想乡都在三维空间里面,所以他们关于距离的定义应该是一样的,实际上对于压缩映射的压缩可以这样表征:
\[
	\symcal d(a,b)> \symcal d(M(a),M(b))
	.\]

这意味着对\textbf{任意}两点 \(a,b\)进行操作之后,他们的象的距离小于原来的距离,这就是压缩。

橙:原来如此,所以单单看压缩这一非常直观的东西可以通过象的距离和源的距离看出来。

蓝:没错,现在这个 \(M\) 就是一个压缩映射了:
\begin{itemize}\kaiti
	\item 将集合 \(\symbb{G} \)映射到自身。
	\item $\forall  a,b\in\symbb{G} ,\,\symcal d(a,b)> \symcal d(M(a),M(b))$。
\end{itemize}

而压缩映射就有一个非常著名的定理:不动点定理,也就是说,对上面的 \(M\)来说的话,必然有\textbf{唯一}的 \(y\),满足:
\[
	M(y)=y
	.\]

你可以认为地图就是幻想乡本身,那么他也是在幻想乡境内,但这样就远远不止一个不动点了。

莉格露:原来如此,这就是为什么地图上必然有与对应的幻想乡的一点重合的点。

橙:那这和ODE有什么关系?

蓝:等会嘛……但是这个定理看似直观,实际上仍有其他要求,那就是 \(\symbb G\) 要是完备的赋范空间(Banach空间)才行。我们的三维空间可以视为 \(\symbb{R} ^3\),那的确是完备的,但有些可不是哦。

莉格露:原来如此,所以你在有理数空间里面搞个地图是没有这种东西的。

蓝:有理数空间是什么……不过不管了,接下来是另外一点,以上我们对不动点的讨论,依赖于距离,集合的所属关系,空间是否完备的要求上,而这些要求,都是通用的,不仅仅只对我们的三维空间有效。我在这里提一嘴,ODE的唯一解,对应的就是那个不动点哦。

露娜、橙、莉格露:我超,是不动点!

蓝:所以说这种抽象的东西能够讨论到更广泛的范围中去嘛……现在我定义这样一个Banach空间 \(\symbb{F} \):
\begin{itemize}\kaiti
	\item 这个空间的元素 \(f(t)\)是从 \(\symbb{I}  \in \symbb{R} \) 到 \(\symbb{R} \)的连续函数,即定义在 \(\symbb{I} \)上的连续函数。
	\item 这个空间的度量 \(\displaystyle \symcal d(a(t),b(t))= \sup_{t\in \symbb{I}}\left\vert a(t)-b(t) \right\vert \)。
	\item 范数可以有 \( \displaystyle \left\| a(t) \right\|_{\symbb{F} }=\sup_{t\in \symbb{I}}\left\vert a(t) \right\vert\)。
\end{itemize}

接下来证明这个空间是完备的。在这里由于时间原因就不证了,可以通过构造Cauthy序列然后用三角不等式证明。然后就是那个著名的\textbf{压缩映射}了。我们知道, \(\symbb{F} \)空间上的点都是一个从 \(\symbb{I}\) 到 \(\symbb{R} \)的连续函数,那么我们需要定义一个对函数的映射。现在我们要紧的是,这个映射和ODE有什么关系。

琪露诺:不知道。

蓝:看下ODE的形式:
\[
	\dx=f(t,x),\, x(t_0)=x_0,\, t,t_0 \in \symbb{I}
	.\]

莉格露:求导算符倒是一个对函数的映射,但是我不太清楚它有什么好的压缩性质。

蓝:你想想看,映射既然是要对应 \( M(y) = y\),那么我们可以尝试把主元分离出来,ODE中的主元 \(x\)一个在 \(f\) 的参数列表里面,一个在求导算符里面,请问怎么分离出来捏?

琪露诺:是积分!

蓝:不错的选择,那接下来就变成了:
\[
	x= \int  f(t,x) \,\mathrm{d}t
	.\]

然后捏?

露娜:嗯……还是很抽象啊,这个积分是不定的……

蓝:对,这是一个问题,我们应该怎么样把它变成确定的呢?

露娜:规定积分的上下限。

蓝:你看看一开始的ODE还有些什么东西。

琪露诺,露娜:是初始条件!

蓝:没错,怎么代进去呢?

琪露诺:呃……我试试……
\[
	x= \int_{t_0}^t f(s,x) \,\mathrm{d}s + x_0
	.\]

也许是这个吧……既保留了积分的形状,初始条件代进去左右两边也相等。

蓝:好。那么我们就认为我们要的那个该死的压缩映射是:
\[
	\symscr{P} (x) :\symbb{F} \mapsto \symbb{F} =\int_{t_0}^t f(s,x) \,\mathrm{d}s + x_0,\, t,t_0\in \symbb{I}
	.\]

注意到积分后依然是定义在 \(\symbb{I}\)上的连续函数,所以这个映射将 \(\symbb{I}\)映为自身,接下来要做的是证明其压缩性质:
\[
	\symcal d(\symscr{P} (x_1),\symscr{P} (x_2))< \symcal d(x_1,x_2)
	.\]

还记得我们定义在 \(\symbb{F} \)空间上的距离吗?
\[
	\symcal d(a(t),b(t))= \sup_{t\in \symbb{I}}\left\vert a(t)-b(t) \right\vert
	.\]

所以这实际上意味着:
\[
	\sup_{t\in \symbb{I}} \left\vert \int_{t_0}^t f(t,a)-f(t,b)\,\mathrm{d}s  \right\vert < \sup_{t\in \symbb{I}} \left\vert a-b \right\vert
	.\]

要始终记住 \(a,b\) 都是 \(t\)的连续函数而不是常数哦,我这里为了看起来不要那么多括号就先不写了。

琪露诺:这都啥啊……这怎么可能证的出来?\(\symbb{I}\)又是什么啊呜呜……

蓝:\(\symbb{I}\)是一开始我们定下的 \(t\)的范围,但实际上这个式子确实太广泛了,因此我们需要新的条件,这个条件可以有很多,我们可以尝试下Lipschitz条件:
\begin{itemize}
	\item 为了方便我们不妨限制 \(\symbb{F} \)里面函数的值域,由于那些函数必然是连续的,那我的 \(\symbb{I}\) 的长度足够小时,函数值域的长度也会任意小,不妨设函数值域是 \(\symbb{J}\)。
	\item 接下来我声明函数 \(f\) 在 \(\symbb{I}\times \symbb{J}\)上面满足,对任意的 \((t,x_1),(t,x_2)\in\symbb I\times\symbb J\),都有存在一个常数 \(L>0\)满足:
	      \[
		      \left\vert  f(t,x_1)-f(t,x_2) \right\vert \leqslant  L \cdot \left\vert x_1-x_2 \right\vert
		      .\]
\end{itemize}
这就是Lipschitz条件啦!

琪露诺:看起来超级怪……

蓝:确实,这是一种比较强的连续形式。如果满足了这一点的话我们能怎么办呢话说?

琪露诺:代进那个方程 \(\sup_{t\in \symbb{I}} \left\vert \int_{t_0}^t f(t,a)-f(t,b)\,\mathrm{d}s  \right\vert < \sup_{t\in \symbb{I}} \left\vert a-b \right\vert\)里,我试试:
\[
	\sup_{t\in \symbb{I}} \left\vert \int_{t_0}^t f(t,a)-f(t,b)\,\mathrm{d}s  \right\vert<L \cdot \sup_{t\in \symbb{I}} \left\vert \int_{t_0}^t (a-b)\,\mathrm{d}s  \right\vert<L \left\vert t-t_0 \right\vert \cdot \sup_{t\in \symbb{I}} \left\vert a-b \right\vert
	.\]

哦哦哦哦哦哦所以只需要 \(L \left\vert t-t_0 \right\vert <1\)就可以了吧。

露娜:原来如此,由于 \(t,t_0\)都局限在 \(\symbb{I}\)上面的,我只要把这个 \(\symbb{I}\)缩小到足够小,那么我就可以保证\(L \left\vert t-t_0 \right\vert <1\)。

蓝:没错,而且缩小\(I\)反而同样会使 \(L\)不增。因此我们可以认为当 \(I\)足够小的时候,\(\symscr{P}\)是一个在完备赋范空间上面的压缩映射,接下来我们只需要利用Banach不动点定理:必然\textbf{唯一存在}一个不动点(实际上就是定义在 \(I\) 上的连续函数)满足:
\[
	\symscr{P} (\symcal X)=\symcal X \implies\int_{t_0}^t f(t,\symcal X) \,\mathrm{d}s + x_0=\symcal X
	.\]

两边求导就得到了 \(\symcal X\)是这个ODE解。

莉格露:真是巧妙的证明!

蓝:事实上,由于这个东西的存在(适当选取超级小的 \(\symbb{I}\)的情况下):
\[
	\symcal d(\symscr{P} (x_1),\symscr{P} (x_2))< \frac{1}{2}\symcal  d(x_1,x_2)
	.\]

我们知道有:
\[
	\symcal d(\symscr{P}^n (x_1),\symscr{P}^n (x_2))< \frac{1}{2}\symcal  d(\symscr{P}^{n-1}(x_1),\symscr{P}^{n-1}x_2)< \cdots < \frac{1}{2^n}\symcal d(x_1,x_2) \to 0
	.\]

这意味着,如果我把两个函数 \(x_1,x_2\)应用 \(\symscr{P} \)操作多次,他们之间的距离会趋向于0,由于我们可以在 \(\symbb{F} \)里面取极限,那么实际上就是他们会趋向于同一个函数,至于为什么距离为0就是同一个函数嘛……嘿嘿嘿……这个极限只可能是ODE的解 \(\symcal X\),因为我对 \(\symcal X\) 应用多少次 \(\symscr{P} \)都是 \(\symcal X\)。

这就给我们一个新的思考:我们可以通过给任意一个函数应用 \(\symscr{P} \)多次来逼近解 \(\symcal X\),实际上,Picard就是这么做的。

此时我们会将这个不动点称为“吸引子”。

琪露诺:这个多次用 \(\symscr{P} \)去逼近解……只有笨蛋才会这么干吧……积分会杀死所有妖怪和人类的。

蓝:当做计算练习许也不错(笑)。

另外,考虑这个问题,我们知道为了证明压缩性,可以尝试:
\[
	\dfrac{\symcal d(\symscr{P} (x_1),\symscr{P} (x_2))} {\symcal  d(x_1,x_2)}< 1,\,\mathrm{where}\ x_1\neq x_2
	.\]

这个让你想到了什么?

露娜:嗯……导数?

蓝:是导数的绝对值,事实上我们考虑一开始的例子:\(x=\cos x\),我们知道在解的附近:
\[
	\frac{\cos x_1-\cos x_2}{x_1-x_2}=\sin \theta <1,\,\theta \in \left( \min\left\{ x_1,x_2 \right\},\max \left\{ x_1,x_2 \right\}   \right)
	.\]

不能等于一哦!因为这个解离导数等于正负一的点挺远的,我只需要把握一下“附近”的程度即可。那么实际上根据压缩映射原理,\( \widehat{\symscr{P}}(x)=\cos x\)同样有唯一的不动点,而且只要对着任何一个 \(x\) 用 \(\widehat{\symscr{P} }\)作用多次,也会收敛到不动点,这就解决了为什么
\[
	\,x_0=0,\,x_n=\cos  x_{n-1}
	.\]

这个序列的极限是方程的解了。